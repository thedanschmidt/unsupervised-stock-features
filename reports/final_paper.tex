\documentclass{article}

% if you need to pass options to natbib, use, e.g.:
% \PassOptionsToPackage{numbers, compress}{natbib}
% before loading nips_2016
%
% to avoid loading the natbib package, add option nonatbib:
% \usepackage[nonatbib]{nips_2016}

\usepackage[final]{nips_2016}

% to compile a camera-ready version, add the [final] option, e.g.:
% \usepackage[final]{nips_2016}

\usepackage[utf8]{inputenc} % allow utf-8 input
\usepackage[T1]{fontenc}    % use 8-bit T1 fonts
\usepackage{hyperref}       % hyperlinks
\usepackage{url}            % simple URL typesetting
\usepackage{booktabs}       % professional-quality tables
\usepackage{amsfonts}       % blackboard math symbols
\usepackage{nicefrac}       % compact symbols for 1/2, etc.
\usepackage{microtype}      % microtypography

\title{Unsupervised Learning of Stock Market Price Features}

% The \author macro works with any number of authors. There are two
% commands used to separate the names and addresses of multiple
% authors: \And and \AND.
%
% Using \And between authors leaves it to LaTeX to determine where to
% break the lines. Using \AND forces a line break at that point. So,
% if LaTeX puts 3 of 4 authors names on the first line, and the last
% on the second line, try using \AND instead of \And before the third
% author name.

\author{
  Dan Schmidt \thanks{Use footnote for providing further
    information about author (webpage, alternative
    address)---\emph{not} for acknowledging funding agencies.} \\
  Department of Mathematics\\
  Harvey Mudd College\\
  Claremont, CA 91711 \\
  \texttt{dschmidt@hmc.edu} \\
}

\begin{document}
% \nipsfinalcopy is no longer used

\maketitle

\begin{abstract}
The traditional research process for financial quants is to construct market features based on intuition about market
function. These features are then engineered for predictive power. In this paper, unsupervised learning methods are
explored to learn stock market price features. SVD, autoencoder, (generative adversial network ?). The resulting features
are tested for predictive power both with a simple linear model and a neural network regression against future returns on
different time scales. (findings). 
\end{abstract}

\section{Introduction}

This paper investigates several different unsupervised methods for learning
features in stock market price data. First, linear models are implemented
using PCA as a baseline comparison. Then neural networks for the purposes
of dimensionality reduction are introduced. The problem is posed as going
from a window of minute price observations to predictions forward on that 
price. The connection between single-layer autoencoders and PCA is made
clear, and then non-linear autoencoder models are applied to try to 
find latent models of price movements over time. Various methods
are tried to come up with a list of feature sets for testing. \\ \\
Once the feature sets are generated, they are tested for predictive
power by fitting them against future returns with 1) a linear model
and 2) a neural model. Individual stocks are tried, along with residual
stock movements (market neutral) and baskets of clustered stocks.

\section{The Data}

\section{Linear Models: PCA}

\section{Autoencoders}

\section{Fitting Future Returns}

\section{Results}

\end{document}
